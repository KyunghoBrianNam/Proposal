\chapter{Introduction}

Article 25 of the Convention on the Rights of Persons
with Disabilities, adopted by the General Assembly of the United
Nations, establishes the right to receive the highest feasible
standard of health care without discrimination on the basis of
disability from State Parties \cite{UN}. Concerning amputees, the
provision of affordable and high-quality prosthetic aids is a human
rights argument, one that governments must support. Nevertheless,
World Health Organisation figures show that only 5-15\% of the
population in need is able to access prosthetic and orthotic
appliances in today's world \cite{WHO}. The difficulty of accessing
such devices is greater in low- and middle-income countries. In such
areas, charities, often staffed by people who are not trained
professionals, often provide prosthetic and orthotic services, leading
to compromised services and poor quality and fit. In addition, without
adequate provision for maintenance, people are often restricted in
their quality of life, excluded from participating in society, and
locked into poverty and isolation \cite{WHO}. Most amputees who have
any kind of prosthetic hand at all are only able to obtain a cosmetic
hand, which affords a realistic appearance \cite{3D4} but provides
little functionality for grasping objects or anything else.

One of the main issues for useful prostheses is cost. The total mean
life cost for an amputee is reported to be \$509,275
\cite{cost_1}. The weighted average prosthesis price is \$10,232, and
total costs for the prosthesis after the second year are \$181,500
\cite{cost}. The annual maintenance costs for a functional prosthesis
are expected to be 20\% of the initial prosthesis' price, for a total
of \$83,490 after the second year.

Most studies in the field of prosthetics have focused on reducing
acquisition costs and improving performance.  However, maintenance
service for prostheses has not been nearly as heavily studied.
Amputees with prostheses certainly require such ongoing service,
especially as new situations arise.  For instance, child users grow up
and need larger-sized prostheses, and old prosthetics suffer decreased
performance.  To address the necessity of excellent quality
maintenance service and the lack of low-cost, high-performing
prostheses, this paper introduces an architecture that can be
self-trainable and user-customized, based on a very low-cost
3D-printed prosthetic hand. This architecture provides self-trainable
software to the user, which can increase the prosthetic hand's
performance at a very low financial cost. A small amount of time
invested in training the user's prosthesis returns high accuracy in
its performance with repetitive behavior.
