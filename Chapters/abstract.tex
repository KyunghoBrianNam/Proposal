\begin{abstractpage}


3D printed prosthetics have narrowed the gap between the tens of
thousands of dollars cost of traditional prosthetic designs and
amputees' needs. However, the World Health Organization estimates that
only 5-15\% of people can receive adequate prosthesis services
~\cite{WHO}. To resolve the lack of prosthesis supply and reduce cost
issues (for both materials and maintenance), this paper provides an
overview of a self-trainable user-customized system architecture for a
3D printed prosthetic hand to minimize the challenge of accessing and
maintaining these supporting devices. In this paper, we develop and
implement a customized behavior system that can generate any gesture
that users desire. The architecture provides upper limb amputees with
self-trainable software and can improve their prosthetic performance
at almost no financial cost.  All kinds of unique gestures that users
want are trainable with the RBF network using 3 channel EMG sensor
signals with a 94\% average success rate.  This result demonstrates
that applying user-customized training to the behavior of a prosthetic
hand can satisfy individual user requirements in real-life activities
with high performance.



\end{abstractpage}
