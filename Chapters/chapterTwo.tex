\chapter{Literature Review}

\section{3D-printed prostheses}
Currently, only 5-15\% of people who need them have access to
prostheses \cite{WHO}, so much effort and research is going into
expanding the availability of low-cost prosthetics.  Several recent
works \cite{3D4}\cite{3D2} aimed to build a low-cost robotic
prosthetic arm using 3D printing technology with capable and
comfortable design. 3D printing technology suggests a new path to
satisfying amputees' financial needs and various physical conditions,
while also enabling departures from the prosthesis's standard size for
children or veterans with upper limb loss. \cite{3D3} For similar
reasons, in order to minimize the cost of supplying the prosthesis to
more people, a 3D-printed hand was chosen as the basic structure for
this study.

\section{EMG-based human robot interaction}
Electromyography (EMG) \cite{review, EMG} is one of several biological
signals produced by the human body which can be used to predict motor
intentions, in addition to such signals as electrical ultrasound (EOG)
\cite{EOG}, electrical ECG \cite{ECOG}, EEG \cite{EEG_1}, and
brainwaves (MEG) \cite{MEG}. Likewise, many different sensors have
been used to read users' intentions to control their prostheses
\cite{MMG}.  In particular, EMG sensors are available that are both
highly accurate and low cost, and can be used to control diverse
devices. The most attractive aspect of an EMG sensor is that it is a
hands-free sensor that does not require invasive surgery.  EMG
sensors \cite{controlEMG} have been used to control a mouse
\cite{mouse}, a mobile RC car \cite{RCcar}, and even to assemble a
Rubik's cube \cite{rubikcube}.

\section{EEG-based human robot interaction}
Another non-invasive control techniques
that translate imagined hand movements into motor signals use
electroencephalograms (EEGs) \cite{EEG} as Brain-Machine Interfaces.
An EEG-based study work \cite{[???]} developed and validated a neuro-based
method for objectively verifying robot behavior in HRI, and proposed to
detect improper / unexpected / erroneous robot behavior using the
electroencephalogram (EEG) of a human interaction partner.
EEG-based brain-controlled mobile robots can be useful tools for 
severely disabled persons in their daily lives, 
particularly when it comes to assisting them in moving 
voluntarily. \cite{EEG}
This EEG human-robot interaction technique has already been successful
in detecting and classifying simple hand motions. \cite{[???]}



Our research relies on a hand prototype \cite{3D1} which emulates the
structure of a human hand using 5 servo motors and electromyography
(EMG) sensors to detect the electrical potential when muscle
contraction appears.

The novel problem we consider is customizing a cheaply-produced
3D-printed prosthetic device for each diverse user.  A
perfectly fitting prosthesis with high performance for all the
different individuals is impracticable.  Most of the research in the
field of prosthetic hands focused only on reducing cost or improving
performance, while prosthesis maintenance services have not been as
intensely studied.  Adequate after-care maintenance services for
amputees who use the prosthesis are extremely important.  We have
developed an architecture that allows users to train their devices to
have high-performance and creative idiosyncratic behaviors tailored
exactly to their individual use cases.